\documentclass{article}
\usepackage[utf8]{inputenc}
\usepackage[english, swedish]{babel}
\usepackage{amsmath}% http://ctan.org/pkg/amsmath
\usepackage{listings}
\usepackage{graphicx}
\usepackage{comment}
\usepackage{float}
\usepackage{lscape}
\usepackage{csquotes}
\usepackage{multirow}
\usepackage{caption}
\usepackage{hyperref}
\usepackage{svg}

\usepackage{biblatex}
\addbibresource{bibliography.bib}

\usepackage[a4paper,pdftex,bottom=20mm,top=20mm, width=160mm]{geometry} % A4paper margins
\setlength{\parindent}{0pt} % Tar bort indenteringen på paragrafer 
\setlength{\parskip}{1em}
\captionsetup[table]{skip=5pt}


\title{
\includegraphics[scale=0.9]{Bilder/liu_logga.png} \\
\vspace{3cm} \includegraphics[scale=0.2]{Bilder/PUM14_logo.png} \\
\vspace{2.0cm} \textbf{Kravspecifikation} \\
 \endgraf\rule{\textwidth}{.4pt}
  \large \textbf{TDDD96 Kandidatarbete i programvaruutveckling}\\
   }
   

\author{
    \parbox{\textwidth}{%
    \begin{center}
        Philip Batan \\
        Gustav Boberg \\
        Marcus Hedquist \\
        William Janowsky \\
        Lucas Lindahl \\
        Erik Luttu \\
        Gabriel Matsson \\
        Michael Werner 
    \end{center}
    }
}


\date{21 Februari 2025}

\begin{document}
\maketitle

\vfill % Fills the rest of the page to push the table to the bottom

% Add the table at the bottom of the page
\begin{center}
Versionshistorik
\end{center}

% Add the table at the bottom of the page
\begin{table}[H]
\centering
\begin{tabular}{|p{.305\textwidth}|p{.305\textwidth}|p{.305\textwidth}|}
    \hline
    \textbf{Version} & \textbf{Granskad av} & \textbf{Granskningsdatum} \\
    \hline
    0.2.0 & PUM 14 & 14-03-2025 \\
    \hline
\end{tabular}
\end{table}

\newpage
\tableofcontents
\newpage

\section*{Dokumenthistorik}
\begin{tabular}{p{.06\textwidth}|p{.11\textwidth}|p{.30\textwidth}|p{.20\textwidth}|p{.13\textwidth}} 
  \multicolumn{1}{c}{\bfseries Version} & 
  \multicolumn{1}{|c}{\bfseries Datum} & 
  \multicolumn{1}{|c}{\bfseries Utförda ändringar} & 
  \multicolumn{1}{|c}{\bfseries Utförda av} & 
  \multicolumn{1}{|c}{\bfseries Granskad}\\
  \hline
  \hline
  0.1.0 & 28-01-2025 & Dokument skapat & Michael Werner & PUM 14 \\
  \hline
 0.1.0 & 13-03-2025 & Fixat alla referenser & Michael Werner & PUM 14 \\
  \hline
\end{tabular}
\newpage

% Content for the next sections starts here
\section{Inledning}
För att effektivisera vårdpersonalens arbete och förbättra översikten av patientjournaler utvecklas systemet Interaktiv visualisering av patientjournaler (IVPJ). Genom en användarvänlig och responsiv webbapplikation ska IVPJ möjliggöra snabb navigering och visning av journalanteckningar, både i listformat och tidslinjeform.

\subsection{Syfte}
Detta dokuments syfte är att fastställa de krav som ställs på systemet 
\textit{Interaktiv visualisering av patientjournaler} (IVPJ).  
Dokumentet riktar sig främst till projektets kund och dess projektgrupp - PUM14. Genom att tydligt lista och beskriva funktionella och icke-funktionella krav ska detta dokument fungera som en gemensam utgångspunkt för vidare design, utveckling och testning.

\subsection{Omfattning}
IVPJ är en webbaserad frontend-applikation som syftar till att ge vårdpersonal en snabb och responsiv översikt av patientjournaler genom ett interaktivt och strukturerat gränssnitt. Systemet ska minimera antalet interaktioner och möjliggöra effektiv navigering för att snabbt nå relevant journaldata.

Projektet omfattar utveckling och testning i en kontrollerad testmiljö med fiktiva data.  
Det kommer därför \textbf{inte} att inkludera hantering av känsliga patientuppgifter och avancerad säkerhetsfunktionalitet. 

\subsection{Definitioner, akronymer, och förkortningar}

\begin{itemize}
    \item \textbf{IVPJ} - Interaktiv visualisering av patientjournaler, systemet som ska utvecklas.
    \item \textbf{API} - Applikationsprogrammeringsgränssnitt, från engelskans 'Application Programming Interface'. Ett gränssnitt mellan applikation och databibliotek. 
    \item \textbf{REST-API} - Representational State Transfer Application Programming Interface, ett standardiserat gränssnitt som möjliggör kommunikation mellan klient och server genom HTTP-begäranden.
    \item \textbf{Frontend} - Den del av ett system som körs i användarens webbläsare.
    \item \textbf{Backend} - Den del av ett system som hanterar logik, databehandling och kommunikation med databaser eller externa system.
    \item \textbf{UI} - Den visuella och interaktiva delen av ett systems frontend som användaren interagerar med.
    \item \textbf{TakeCare} - Ett journalsystem som används inom vården.
    \item \textbf{OpenEHR} - En standardiserad metod för hantering av patientjournaler, där data organiseras genom en hierarkisk struktur. 
    \item \textbf{HL7 FHIR} - 'Fast Healthcare Interoperability Resources' är en global standard för utbyte av hälso- och sjukvårdsinformation.
    \item \textbf{Svelte} - Ett modernt frontend-ramverk som kompilerar till optimerad JavaScript-kod vid byggtid, vilket resulterar i snabbare rendering och lägre resursförbrukning jämfört med traditionella ramverk.
    \item \textbf{D3.js} - Ett JavaScript-bibliotek för datavisualisering som möjliggör skapande av interaktiva grafer och diagram.
\end{itemize}

\subsection{Referenser}
\renewcommand{\refname}{\vspace{-2em}}
\nocite{720574}
\printbibliography


\subsection{Översikt}
Resterande delar av dokumentet är uppdelade i två sektioner. Den första sektionen är helhetsbeskrivningen, där systemets roll, struktur och huvudsakliga funktioner beskrivs. Här presenteras produktperspektivet, som ger en övergripande bild av hur IVPJ samverkar med andra system. Därefter följer en genomgång av produktfunktionerna, där systemets komponenter och deras kopplade funktioner beskrivs.

Vidare innehåller helhetsbeskrivningen en sektion om användarkaraktäristik, där de primära användarna identifieras och deras tekniska kompetens och erfarenhet beskrivs. Därefter beskrivs projektets begränsningar som klargör dess omfattning och tekniska avgränsningar, samt antaganden och beroenden, där de externa faktorer som systemet förutsätter listas.

Den sista delen av dokumentet, specifika krav, innehåller detaljerade tekniska och funktionella krav.

\section{Helhetsbeskrivning}
Helhetsbeskrivningen består av ett produktperspektiv, beskrivning av produktfunktioner, gränssnittsbeskrivning, användarkaraktäristik, begränsningar, antaganden och beroenden. 

\subsection{Produktperspektiv}

\begin{figure}[h]
    \centering
    \includegraphics[width=1\linewidth]{Bilder/produkt_perspektiv_v2.png}
    \caption{IVPJ och dess gränsnitt}
    \label{fig:gr}
\end{figure}

IVPJ är en fristående webbaserad frontend-applikation som kommunicerar med externa system genom REST-API:er. För att säkerställa enhetlighet och kompatibilitet använder IVPJ standardiserade format som openEHR och HL7 FHIR för att hämta och visualisera journaldata.

Systemet drar nytta av moderna frontend-teknologier för att skapa en responsiv och interaktiv användarupplevelse. Ramverket Svelte används för snabb rendering och effektiv resursanvändning, medan bibliotek som D3.js möjliggör avancerade visualiseringar, såsom tidslinjer och grafer. Dessa teknologier stödjer utvecklingen av de funktioner som behövs för att interagera med stora mängder journaldata på ett strukturerat och överskådligt sätt.

IVPJ fungerar som en ren frontend-lösning utan egna backend-komponenter. Applikationen är beroende av externa API:er och standarder för datakommunikation, vilket gör den lätt att integrera med befintliga och framtida vårdsystem. Relationerna mellan applikationen och dess gränssnitt visas i figur \ref{fig:gr}.

\newpage
\subsection{Produktfunktioner}

Systemets funktionalitet är uppdelad i tre huvudområden: användarfunktioner, användargränssnitt och systemfunktioner.

Figur \ref{fig:komp} illustrerar hur de olika områdena i systemet är relaterade till varandra. På den översta nivån finns användarfunktionerna som beskriver den funktionalitet användaren har tillgång till. I nästa nivå finns användargränssnittet som visar hur användarfunktionerna presenteras för användaren. Här delas presentationen upp i två huvudsakliga vyer, basvyn och tidslinjevyn, där en delad filterkomponent gör det möjligt att filtrera vilka journalanteckningar som ska visas i respektive vy. Längst ner finns systemfunktionerna, där systemet hämtar journalanteckningar från externa källor och gör dem tillgängliga för användaren.

\begin{figure}[h]
    \centering
    \includesvg[width=1\linewidth]{komponenter_5.svg}
    \caption{Översikt över IVPJ:s användargränssnittskomponenter och deras relation till systemets funktioner.}
    \label{fig:komp}
\end{figure}

\subsubsection{Användarfunktioner}

\begin{itemize}
    \item \textbf{Visa i lista} \\
    Journalanteckningarnas metadata presenteras i en klassisk listvy, där de sorteras i kronologisk ordning. Listan fungerar som en navigationspunkt där användaren kan hantera journalanteckningar.

    \item \textbf{Visa journalanteckningar i tidslinje} \\
    Journalanteckningarna i sin helhet visas i en interaktiv tidslinje. Varje journalanteckning representeras visuellt längs en tidsaxel, vilket gör det lättare att få en snabb överblick över patientens vårdhistorik.

    \item \textbf{Markera} \\
    Användaren kan markera en eller flera journalanteckningar i listan för att öppna dem i en detaljerad vy.

    \item \textbf{Visa markerade journalanteckningar dynamiskt} \\
    Journalanteckningar anpassar sig efter utrymmet och antalet markerade anteckningar. Om en journal är markerad fyller den mer yta, medan flera journaler arrangeras dynamiskt för optimal läsbarhet.

    \item \textbf{Justera journalanteckningars dynamik} \\
    Användaren kan justera hur markerade journaler visas genom att ändra layout eller utrymmesfördelning i vyn.

    \item \textbf{Låsa journalanteckningar} \\
    Användaren kan låsa en journalanteckning så att den bevaras i vyn även om användaren scrollar bort från den.

    \item \textbf{Bevara markerade/låsta journalanteckningar} \\
    Systemet säkerställer att markerade eller låsta journalanteckningar förblir synliga oavsett vy-byte.

    \item \textbf{Kollapsa och expandera} \\
    Användaren kan kollapsa eller expandera grupper av journalanteckningar i listan för att hantera stora mängder information.

    \item \textbf{Visa minimerade journalanteckningar} \\
    Användaren får en kompakt visning av journalanteckningar som är minimerade på en tidslinje för att få en snabbare översikt.

    \item \textbf{Slider/Scrolla} \\
    Översiktstidlinjen möjliggör horisontell scrollning eller slider-funktionalitet för att navigera i stora mängder journalanteckningar på tidslinjen.

    \item \textbf{Zooma in och zooma ut} \\
    Användaren kan zooma in och ut på tidslinjen för att justera detaljnivån på de visade journalanteckningarna.

    \item \textbf{Filtrera} \\
    Journalanteckningar kan filtreras för att visa endast den mest relevanta informationen.

    \item \textbf{Färgkoda} \\
    För att förbättra läsbarheten kan journaltext färgkodas utifrån nyckelord eller kategorier.
\end{itemize}


\subsubsection{Användargränssnitt}

\begin{itemize}
    \item \textbf{Lista - Översiktlig} \\
    Listan är användargränssnittet för list-funktionalitet som visar journalanteckningar i kronologisk ordning och möjliggör markering.

    \item \textbf{Detaljerad visning} \\
    Den detaljerade visningen presenterar markerade journalanteckningar och anpassar layouten baserat på antalet anteckningar och användarens preferenser.

    \item \textbf{Filter} \\
    Filterkomponenten presenterar filtreringsalternativen för användaren.

    \item \textbf{Tidslinje - Detaljerad} \\
    Den detaljerade tidslinjevyn presenterar journalanteckningar på en tidsaxel med hög detaljrikedom.

    \item \textbf{Tidslinje - Översiktlig} \\
    Den översiktliga tidslinjevyn ger en mer komprimerad visning av journalanteckningar och gör det möjligt att zooma, scrolla för att justera detaljnivå.

    \item \textbf{Basvy} \\
    Basvyn är den primära arbetsytan där användaren interagerar med journalanteckningar.

    \item \textbf{Tidslinjevy} \\
    Tidslinjevyn är en alternativ vy där journalanteckningarna visas längs en detaljerad samt en översiktlig tidsaxel.
\end{itemize}

\subsubsection{Systemfunktioner}

\begin{itemize}
    \item \textbf{Hämta journalanteckningar} \\
    Systemets kärnfunktion är att hantera och hämta journalanteckningar från externa datakällor enligt standardiserade gränssnitt.
\end{itemize}


\newpage
\subsection{Gränssnittsbeskrivning}
Applikationens gränssnitt delas upp i användargränssnitt, mjukvarugränssnitt, hårdvarugränssnitt och kommunikationsgränssnitt. 
\subsubsection{Användargränssnitt}
Användargränssnittet ska vara enkelt att använda men väl anpassningsbart för att möjliggöra den effektivaste möjliga arbetsmiljön för den unika användaren. 

När användaren loggar in i systemet för första gången ska denne bemötas av en start-vy som efterliknar Karolinska Sjukhusets gamla TakeCare system. Användaren ska ha möjlighet att justera hur gränssnittet ser ut genom att skala de olika byggblocken. När användaren är nöjd med sin start-vy ska den kunna välja att spara den som ny start-vy. Om det inte är användarens första inloggning så bemötes denne istället av egen sparad start-vy. Ett exempel på en start-vy som användaren ska kunna ha visas i figur \ref{fig:start-vy}.

\begin{figure}[H]
    \centering
    \includegraphics[width=0.7\linewidth]{Bilder/Exempel Start-vy.png}
    \caption{Start-vy}
    \label{fig:start-vy}
\end{figure}

Genom att interagera med menyn i övre vänstra hörnet i figur \ref{fig:start-vy} ska användaren kunna skräddarsy sin upplevelse av miljön. Denna meny ska vara liten och ur vägen för resten av vyn i syfte att minimera mängden inställningar som tar upp plats på skärmen. En inställning som ska finnas i denna meny är inställningen om färgval för färgfiltrering, såsom illustrerad i figur \ref{fig:färgfilterprofil}. Inställningsmenyn ska dyka upp när användaren klickar på tillhörande ikon och ska vara öppen till och med att användaren klickar bort från menyn eller klickar på ikonen igen. 

\begin{figure}[H]
    \centering
    \includegraphics[width=0.5\linewidth]{Bilder/Färgprofilering.png}
    \caption{Val av färgfilterprofil}
    \label{fig:färgfilterprofil}
\end{figure}

Applikationen ska visa användaren en tidslinje med start- och slutdatum. Start- och slutdatum filtrerar bort de dokument utanför valt tidsintervall. Användaren ska själv kunna välja start- och slutdatum den vill fokusera på, men ska inte kunna välja datum utanför patientens historik. Dessa datum ska presenteras för användaren likt i figur \ref{fig:tidslinje}. Dokument som öppnas ska markeras på tidslinjen med en tydlig färgning såsom grön. Placeringen ska vara relativt dokumentets datum, där det senaste dokumentet på en patient ska ligga längst till höger på tidslinjen, medan det tidigaste ska ligga längst till vänster på tidslinjen.

\begin{figure}[H]
    \centering
    \includegraphics[width=0.7\linewidth]{Bilder/Tidslinje Koncept.png}
    \caption{Tidslinje och tidsmarkering av öppnade dokument}
    \label{fig:tidslinje}
\end{figure}

Alla dokument som skapades inom det valda tidsintervallet ska visas på skärmen i ett format sådant att användaren kan se alla dokument listade utan att behöva klicka in på dem. Vidare ska det även markeras vilka dokument som är öppna, till exempel i formen av en enkel checkbox ruta. Ett exempel på en vy som uppnår dessa krav illustreras i figur \ref{fig:dokument}. 

\begin{figure}[H]
    \centering
    \includegraphics[width=0.7\linewidth]{Bilder/Journallista ex.png}
    \caption{Lista av dokument i journalsystem för godtycklig patient}
    \label{fig:dokument}
\end{figure}



\noindent
Användaren ska kunna öppna så många dokument som den önskar. Vid brist på utrymme placeras dokumentet längre ner i dokumentfönstret och ett scrollhjul dyker upp längs dokumentfönstrets högerkant. Om dokumentet är längre än en sida så skapas dessutom ett scrollhjul för dokumentet. För att skrolla i dokumentet behöver användaren klicka någonstans i dokumentet. För att sedan kunna skrolla i dokumentfönstret klickar man antingen någonstans på skärmen som inte är ett dokument eller sidopanel som också inkluderar ett scrollhjul. Alternativt ska användaren kunna använda en snabbtangent för att kunna åstadkomma samma resultat. Dokumentet ska visa om det är valt. Dokumenten i dokumentfönstret ska dessutom vara fritt flyttbara relativt varandra samt skalbara. Man kan se två exempel på dokument-vyer i figur \ref{fig:dokumentvy1} och figur \ref{fig:dokumentvy2}. 

\begin{figure}[H]
  \centering
  \begin{minipage}{0.45\textwidth}
    \includegraphics[width=\linewidth]
    {Bilder/Dokumentvy Ex 1.png}
    \caption{Dokument-vy exempel 1}
    \label{fig:dokumentvy1}
  \end{minipage}
  \hfill
  \begin{minipage}{0.45\textwidth}
    \includegraphics[width=\linewidth]
    {Bilder/Dokumentvy Ex 2.png}
    \caption{Dokument-vy exempel 2}
    \label{fig:dokumentvy2}
  \end{minipage}
\end{figure}

\noindent
Om användaren klickar på tidslinjekomponenten eller en knapp i övre vänstra hörnet ska vyn förändras till en dynamisk tidslinje-vy som illustreras i figur \ref{fig:tidslinjevy}. Vyn ska inkludera utdrag av alla dokument sorterade efter datum inom givet intervall i ett format som ger en snabb inblick i dokumentens innehåll. Denna vy är endast visuellt distinkt från basvyn, och ska därmed kunna förmedla samma information. Detta inkluderar tidsintervall, filter, färgning samt redan öppnade dokument. Noterbart ska dokument öppnade i start-vyn fortfarande vara tydligt distinkta i tidslinje-vyn. I illustrationen visas ett exempel på hur man kan visuellt förmedla öppnade dokument via skalning. 
\begin{figure}[H]
    \centering
    \includegraphics[width=1\linewidth]{Bilder/Tidslinje vy.png}
    \caption{Tidslinje-vy}
    \label{fig:tidslinjevy}
\end{figure}

\subsubsection{Hårdvarugränssnitt}
Systemets hårdvarugränssnitt tillhör inte projektets omfattning och sköts av systemets redan existerande infrastruktur. Hårdvaru kopplingen till databasen sköts dessutom av det underliggande operativsystemet. 

\subsubsection{Mjukvarugränssnitt}
Applikationen ska vara kompatibel med den större OpenEHR platformen på Karolinska Universitetssjukhuset. Vid utökning av projekt ska applikationen dessutom använda FHIR för att representera data i graf-, figur- och tabellformat. 

\subsubsection{Kommunikationsgränssnitt}
Kommunikationen mellan databas och applikation ska föras genom en REST-API. Resterande kommunikation sköts av underliggande operativsystem. 
\newpage
\subsection{Användarkaraktäristik}
Systemet kommer användas av primärt vårdgivare, kliniker och forskare. 
\begin{itemize}
    \item \textbf{Utbildningsnivå}: Användare har gymnasieutbildning och löpande medicinsk- och/eller vårdutbildning. 
    \item \textbf{Erfarenhet}: Användaren har vana att arbeta med Karolinska Sjukhusets TakeCare system eller liknande. 
    \item \textbf{Teknisk kompetens}: Användaren förväntas ha grundläggande datorvana. Detta inkluderar att kunna navigera en grundläggande hemsida; justera storleken på ett fönster med hjälp av muspekaren; skrolla igenom en sida med hjälp av bland annat scrollhjulet; med mera. 
\end{itemize}

\subsection{Begränsningar}
Utvecklingen av IVPJ är begränsad till att enbart omfatta frontend-applikationen, vilket innebär att systemet inte inkluderar någon egen backend-lösning. All data hämtas från externa journalsystem via REST-API:er och IVPJ kan därför inte lagra eller bearbeta journaldata utanför det som tillhandahålls av dessa system.  

IVPJ är en webbaserad applikation och fungerar endast på moderna webbläsare, vilket kan innebära begränsat stöd för äldre webbläsare eller andra plattformar som inte följer aktuella webbstandarder.

\subsection{Antaganden och Beroenden}
Utvecklingen av IVPJ bygger på ett antal antaganden och beroenden som påverkar systemets funktionalitet och integration med befintliga vårdsystem. Ett grundläggande antagande är att journalsystemen tillhandahåller fungerande REST-API:er som följer openEHR-standarden.

Vidare antas att användarna av systemet har tidigare erfarenhet av liknande journalsystem, såsom TakeCare, och därmed en grundläggande förståelse för hur digitala verktyg används inom vården.

Eftersom IVPJ endast är en frontend-applikation och utvecklas i en testmiljö utan hantering av känsliga patientuppgifter, förutsätts det att befintliga vårdsystem ansvarar för nödvändig säkerhetshantering, inklusive autentisering, auktorisering och skydd av patientdata.

Då IVPJ är en webbaserad applikation förutsätter den tillgång till en stabil internetuppkoppling. Systemets realtidsfunktionalitet och möjligheten att hämta uppdaterad journaldata bygger på att nätverksanslutningen är tillräckligt snabb och stabil för att hantera API-anrop utan fördröjningar.

\newpage
\section{Specifika krav}
De specifika kraven är uppdelade i externa gränssnittskrav, funktionella krav och prestationskrav samt designbegränsningar. Krav som inte tillhör någon av ovanstående kategorier specificeras i kapitel 3.5.

Varje krav tilldelas ett kravnummer betecknat K[Y]-[X] där [Y] noterar kategori (Se Figur \ref{fig:komp} för indelningen av kategorier) och [X] noterar ordning inom kravkategori. Varje krav tilldelas dessutom en prioritet, där ett prioritet 1 krav har högst prioritet och ett prioritet 3 krav har lägst prioritet. Om ett krav beror på att ett eller flera andra krav är uppfyllda så inkluderas deras kravnummer i kravets beroenden. \\ 

\begin{table}[h]
    \caption{Exempel på funktionella krav i tabell}
    \label{tab:placeholder_label}
    \centering
    \begin{tabular}{|p{0.1\linewidth}|p{0.1\linewidth}|p{0.4\linewidth}|p{0.09\linewidth}|p{0.12\linewidth}|p{0.09\linewidth}|}
        \hline
        \textbf{Krav. nr} & \textbf{Namn} & \textbf{Kravbeskrivning} & \textbf{Prioritet} & \textbf{Beroenden}  & \textbf{Version} \\ \hline
        K1-0   & Färgning av text & Text ska kunna färgläggas baserat  på patientdata.   & 1 & & 0.1.0 \\ \hline
        K1-1   & Filtrering av text   & Text ska kunna filtrera viktiga stycken baserat på färg. & 1 & K2-0 & 0.1.0 \\ \hline
    \end{tabular}
\end{table}

\subsection{Externa gränssnittskrav}
Denna sektion beskriver de krav som relaterar till hur systemet interagerar med externa system och användare. Projektet behandlar endast krav relaterade till användargränssnittet.

\subsubsection{Användargränssnitt}
Denna sektion specificerar krav på användargränssnittet, vilket beskriver hur information presenteras för användaren. 

\begin{table}[H] 
    \caption{Basvy}
    \label{tab:anvandargranssnitt}
    \centering
    \begin{tabular}{|p{0.1\linewidth}|p{0.1\linewidth}|p{0.4\linewidth}|p{0.09\linewidth}|p{0.12\linewidth}|p{0.09\linewidth}|}
        \hline
        \textbf{Krav. nr} & \textbf{Namn} & \textbf{Kravbeskrivning} & \textbf{Prioritet} & \textbf{Beroenden}  & \textbf{Version} \\ \hline
        K1.1-1 & Lista - Översiktlig & Basvyn ska inkludera en dokumentlista. & 1 & K1.1-4 & 0.1.0 \\ \hline
        K1.1-2 & Detaljerad visning & En detaljerad vy ska finnas för att visa markerade journalanteckningar i basvyn. & 1 & K1.1-4 & 0.1.0\\ \hline
        K1.1-3 & Filter & Basvyn ska innehålla ett filter. & 1 & K1.1-4 &\\ \hline
        K1.1-4 & Bavsy & Det ska existera en basvy. & 0 & & 0.1.0\\ \hline
    \end{tabular}
\end{table}

\begin{table}[H] 
    \caption{Tidslinjevy}
    \label{tab:anvandargranssnitt}
    \centering
    \begin{tabular}{|p{0.1\linewidth}|p{0.1\linewidth}|p{0.4\linewidth}|p{0.09\linewidth}|p{0.12\linewidth}|p{0.09\linewidth}|}
        \hline
        \textbf{Krav. nr} & \textbf{Namn} & \textbf{Kravbeskrivning} & \textbf{Prioritet} & \textbf{Beroenden}  & \textbf{Version} \\ \hline
        K2.1-1 & Tidslinje - Detaljerad & En detaljerad tidslinje ska finnas där journalanteckningar visas fullständigt på en tidsaxel. & 1 & K2.1-4 & 0.1.0\\ \hline
        K2.1-2 & Tidslinje - Översiktlig & En förenklad tidslinje ska finnas där en sammanfattad version av journalanteckningarna presenteras. & 1 & K2.1-4 & 0.1.0\\ \hline
        K2.1-3 & Filter & Tidslinjevyn ska innehålla ett filter. & 1 & K2.1-4 & 0.1.0\\ \hline
        K2.1-4 & Tidslinjevy & Det ska existera en tidslinjevy. & 0 & & 0.1.0\\ \hline
    \end{tabular}
\end{table}

\newpage
\subsection{Funktionella krav}
Denna sektion beskriver de funktionella kraven på systemet, det vill säga de specifika funktioner som systemet måste uppfylla för att användaren ska kunna interagera med det på avsett sätt. Funktionella krav delas in i användarfunktioner och systemfunktioner.

\subsubsection{Användarfunktioner}
Denna sektion beskriver de funktioner som är direkt tillgängliga för användaren.

\begin{table}[H]
    \caption{Funktionella krav för användarfunktioner inom basvyn.} 
    \label{tab:användar_funktioner}
    \centering
    \begin{tabular}{|p{0.1\linewidth}|p{0.1\linewidth}|p{0.4\linewidth}|p{0.09\linewidth}|p{0.12\linewidth}|p{0.09\linewidth}|}
        \hline
        \textbf{Krav. nr} & \textbf{Namn} & \textbf{Kravbeskrivning} & \textbf{Prioritet} & \textbf{Beroenden}  & \textbf{Version} \\ \hline
        K1.2-1 & Justera JA dynamik & Användaren ska kunna justera hur markerade journalanteckningar visas. & 2 & K1.2-2 & 0.1.0\\ \hline
        K1.2-2 & Visa markerade JA dynamiskt & Markerade journalanteckningar ska anpassa sin visning för att optimera skärmytan. & 1 & K1.1-2 & 0.1.0\\ \hline
        K1.2-3 & Markera JA & Användaren ska kunna markera en eller flera journalanteckningar från listan för att öppnas i detaljvy. & 1 & K1.2-5 & 0.1.0\\ \hline
        K1.2-4 & Kollapsa \& expandera & Användaren ska kunna kollapsa och expandera grupper av journalanteckningar i listan för att hantera stora mängder data. & 2 & K1.2-5 & 0.1.0\\ \hline
        K1.2-5 & Visa i lista & Journalanteckningar ska kunna presenteras i en kronologisk listvy. & 1 & K1.1-1, K3.2-3 & 0.1.0\\ \hline
    \end{tabular}
\end{table}

\begin{table}[H]
    \caption{Funktionella krav för användarfunktioner inom tidslinjevyn.} 
    \label{tab:användar_funktioner}
    \centering
    \begin{tabular}{|p{0.1\linewidth}|p{0.1\linewidth}|p{0.4\linewidth}|p{0.09\linewidth}|p{0.12\linewidth}|p{0.09\linewidth}|}
        \hline
        \textbf{Krav. nr} & \textbf{Namn} & \textbf{Kravbeskrivning} & \textbf{Prioritet} & \textbf{Beroenden}  & \textbf{Version} \\ \hline
        K2.2-1 & Låsa JA & Användaren ska kunna låsa journalanteckningar i tidslinjen så de ej lämnar skärmen. & 2 & K2.2-4 &0.1.0 \\ \hline
        K2.2-2 & Slider/ Scrolla & Användaren ska kunna scrolla horisontellt eller använda en slider för att navigera i tidslinjevyn. & 2 & K2.2-4, K2.2-5 & 0.1.0\\ \hline
        K2.2-3 & Zooma in \& ut & Användaren ska kunna zooma in och ut på tidslinjen. & 2 & K2.2-5 & 0.1.0\\ \hline
        K2.2-4 & Visa JA i tidslinje & Journalanteckningar ska visas i detalj i en tidslinje. & 1 & K2.1-1, K3.2-3 & 0.1.0\\ \hline
        K2.2-5 & Visa minimerade JA & Användaren ska kunna utnyttja en minimerad tidslinje för bättre överblick. & 2 & K2.1-2, K3.2-3 & 0.1.0\\ \hline
    \end{tabular}
\end{table}

\begin{table}[H]
    \caption{Funktionella krav för användarfunktioner i delad funktionalitet.} 
    \label{tab:användar_funktioner}
    \centering
    \begin{tabular}{|p{0.1\linewidth}|p{0.1\linewidth}|p{0.4\linewidth}|p{0.09\linewidth}|p{0.12\linewidth}|p{0.09\linewidth}|}
        \hline
        \textbf{Krav. nr} & \textbf{Namn} & \textbf{Kravbeskrivning} & \textbf{Prioritet} & \textbf{Beroenden}  & \textbf{Version} \\ \hline
        K3.2-1 & Bevara markerade/låsta JA & Systemet ska säkerställa att markerade och låsta journalanteckningar förblir synliga även vid vybyte. & 2 & K1.2-3, 2.2-1 & 0.1.0\\ \hline
        K3.2-2 & Färgkoda JA & Journalanteckningstext ska färgkodas baserat på nyckelord eller kategorier. & 2 & K1.2-2, K2.2-4, K3.2-3 & 0.1.0\\ \hline
        K3.2-3 & Filtrera & Användaren ska kunna filtrera journalanteckningar. & 1 & K1.1-3, K2.1-3 & 0.1.0\\ \hline
        K3.2-4 & Filtrera tidsintervall & Användaren ska kunna filtrera baserat på ett valt tidsintervall. & 1 & K3.2-3 & 0.1.0\\ \hline
        K3.2-5 & Filtrera journaltyp & Användaren ska kunna filtrera efter journaltyp. & 1 & K3.2-3 & 0.1.0\\ \hline
        K3.2-6 & Filtrera vårdenhet & Användaren ska kunna filtrera baserat på vårdenhet eller avdelning. & 1 & K3.2-3 & 0.1.0\\ \hline
        K3.2-7 & Filtrera sökord & Användaren ska kunna filtrera baserat på sökord. & 1 & K3.2-3 & 0.1.0\\ \hline
        K3.2-8 & Filtrera yrkesroll & Användaren ska kunna filtrera baserat på yrkesroll. & 1 & K3.2-3 & 0.1.0\\ \hline
        K3.2-9 & Kombinera filter & Användaren ska kunna använda flera filter samtidigt. & 2 & K3.2-3 & 0.1.0\\ \hline
        K3.2-10 & Återställa filter & Användaren ska kunna återställa filtreringen till standardinställningar med ett knapptryck. & 2 & K3.2-3 & 0.1.0\\ \hline
        K3.2-11 & Filtrering i realtid & Journalanteckningar ska uppdateras dynamiskt när användaren ändrar filtreringskriterier. & 2 &K3.2-3 & 0.1.0\\ \hline
        K3.2-12 & Spara filter & Användaren ska kunna spara och återanvända ofta använda filterinställningar. & 2 & K3.2-3 & 0.1.0\\ \hline
        K3.2-13 & Färgval & Användaren ska kunna välja färg som används vid färgkodning. & 2 & K3.2-2 & 0.1.0\\ \hline
        K3.2-14 & Spara Layouts & Användaren ska kunna spara önskad layout av systemet. & 2 & K1.1-4, K2.1-4 & 0.1.0\\ \hline
        K3.2-15 & Panel-skalning & Användaren ska kunna skala dokument och sidopaneler. & 2 & K1.1-4, K2.1-4 & 0.1.0\\ \hline
        K3.2-16 & Vy navigering & Användaren ska kunna navigera mellan Basvyn och Tidslinjeyn & 1 & K2.1-4, K1.1-4 & 0.1.0\\ \hline
    \end{tabular}
\end{table}

\newpage
\subsubsection{Systemfunktioner}
Denna sektion beskriver funktioner som är kritiska för systemets tekniska funktionalitet men som inte direkt involverar användarens interaktion.

\begin{table}[H]
    \caption{Funktionella krav för systemfunktioner.} 
    \label{tab:systemfunktioner} 
    \centering
    \begin{tabular}{|p{0.1\linewidth}|p{0.1\linewidth}|p{0.4\linewidth}|p{0.09\linewidth}|p{0.12\linewidth}|p{0.09\linewidth}|}
        \hline
        \textbf{Krav. nr} & \textbf{Namn} & \textbf{Kravbeskrivning} & \textbf{Prioritet} & \textbf{Beroenden}  & \textbf{Version} \\ \hline
        K3.3-1 & Hämta journaldata & Systemet ska hämta journalanteckningar från externa journalsystem via REST-API:er. & 1 & & 0.1.0\\ \hline
        K3.3-2 & Stöd för openEHR & Systemet ska stödja openEHR-standarden. & 1 & K3.3-1 & 0.1.0\\ \hline
        K3.3-3 & Stöd för HL7 FHIR & Systemet ska stödja HL7 FHIR-standarden. & 2 & K3.3-1 & 0.1.0\\ \hline
        K3.3-4 & Uppdatera journaldata i realtid & Systemet ska kunna uppdatera journalanteckningar utan att användaren behöver ladda om sidan. & 2 & K3.3-1 & 0.1.0\\ \hline
        K3.3-5 & Hantera fel vid datahämtning & Systemet ska kunna identifiera och hantera fel vid hämtning av journaldata, exempelvis genom att visa en felmeddelande. & 2 & K3.3-1 & 0.1.0\\ \hline
    \end{tabular}
\end{table}

\subsection{Prestationskrav}
Här detaljeras krav med avseende på systemets prestation. Detta berör systemets tekniska krav, såsom hastighet, tillgänglighet och responstid m.m. 

\begin{table}[H]

    \caption{Prestationskrav}
    \label{tab:placeholder_label}
    \centering
    \begin{tabular}{|p{0.1\linewidth}|p{0.1\linewidth}|p{0.4\linewidth}|p{0.09\linewidth}|p{0.12\linewidth}|p{0.09\linewidth}|}
        \hline
        \textbf{Krav. nr} & \textbf{Namn} & \textbf{Kravbeskrivning} & \textbf{Prioritet} & \textbf{Beroenden}  & \textbf{Version} \\ \hline
        K4-1 & Snabbt gränssnitt & Gränssnittet ska laddas in på under 1.5 sekunder, mätt genom Google Lighthouse.  & 0 & & 0.1.0\\ \hline
        K4-2 & Hjälp för färgblinda & Gränssnittet ska använda sig av färgblind vänliga färgskalor vid färgning. & 2 & & 0.1.0\\ \hline
    \end{tabular}
\end{table}

\subsection{Designbegränsningar}
Här kartläggs de designbegränsningar som behövs för att överensstämma med projektets tidsram, plattform och syfte. 

\begin{table}[H]

    \caption{Designbegränsningar}
    \label{tab:placeholder_label}
    \centering
    \begin{tabular}{|p{0.1\linewidth}|p{0.1\linewidth}|p{0.4\linewidth}|p{0.09\linewidth}|p{0.12\linewidth}|p{0.09\linewidth}|}
        \hline
        \textbf{Krav. nr} & \textbf{Namn} & \textbf{Kravbeskrivning} & \textbf{Prioritet} & \textbf{Beroenden}  & \textbf{Version} \\ \hline
        K5-1 & Presenter-bar leverans & En godtyckligt presenterbar produkt ska levereras senast innan den 19:e Maj. & 0 & & 0.1.0\\ \hline
        K5-2 & Tidsbudget & Systemet får inte ta mer än 3200 timmar, med en buffert på 320 timmar extra, att färdigställa. & 1 & & 0.1.0\\ \hline
    \end{tabular}
\end{table}

\subsection{Andra krav}
Här detaljeras krav som inte kan kategoriseras under någon av ovanstående delmängder. 

\begin{table}[H]

    \caption{Andra krav}
    \label{tab:placeholder_label}
    \centering
    \begin{tabular}{|p{0.1\linewidth}|p{0.1\linewidth}|p{0.4\linewidth}|p{0.09\linewidth}|p{0.12\linewidth}|p{0.09\linewidth}|}
        \hline
        \textbf{Krav. nr} & \textbf{Namn} & \textbf{Kravbeskrivning} & \textbf{Prioritet} & \textbf{Beroenden}  & \textbf{Version} \\ \hline
        K6-1 & Dokument-ation & All dokumentation ska skrivas på svenska & 1 & & 0.1.0\\ \hline
        K6-2 & Källkod kommentarer & All källkod ska kommenteras på engelska & 1 & & 0.1.0\\ \hline
        K6-3 & Källkod & All källkod ska skrivas på engelska & 1 & & 0.1.0\\ \hline
        K6-4 & Enhetsstöd & Det ska finnas en version av gränssnittet anpassad för, och skalad till, surfplattor och andra touchskärmar. & 0 & & 0.1.0\\ \hline
        K6-5 & Iterativt Arbetsmetodik & Projektet ska använda sig av arbetsmetoden "Iterativ Utveckling" & 1 &  & 0.1.1\\ \hline
        K6-6 & SCRUM & Projektet ska följa ramverket SCRUM. & 1 &  & 0.1.1\\ \hline
    \end{tabular}
\end{table}

\end{document}
