\documentclass[a3paper,landscape]{article}
\usepackage[utf8]{inputenc}
\usepackage[swedish]{babel}
\usepackage{geometry}
\usepackage{longtable}
\usepackage{tabu}
\usepackage{xcolor}
\usepackage{colortbl}
\usepackage{hyperref}
\usepackage{array}
\usepackage{booktabs}
\usepackage{pdflscape}
\usepackage{fancyhdr}
\usepackage{lipsum}

% Minska marginaler till minimum för att maximalt utnyttja A3-formatet
\geometry{
    a3paper,
    landscape,
    left=2mm,
    right=2mm,
    top=5mm,
    bottom=5mm,
}

% Uppdatera kolumntyper för alignment med bättre bredder
\newcolumntype{L}[1]{>{\raggedright\arraybackslash\small}p{#1}}
\newcolumntype{C}[1]{>{\centering\arraybackslash\small}p{#1}}
\newcolumntype{R}[1]{>{\raggedleft\arraybackslash\small}p{#1}}

% Definiera färger för olika typer av tester
\definecolor{unitcolor}{RGB}{230,240,255}
\definecolor{integrationcolor}{RGB}{255,240,230}
\definecolor{systemcolor}{RGB}{230,255,230}
\definecolor{performancecolor}{RGB}{255,230,230}

\title{Testspecifikation för PUM-system}
\author{Grupp 14}
\date{\today}

% Aktivera formulärfunktionalitet i hyperref
\hypersetup{
    pdfauthor={Grupp 14},
    pdftitle={Testspecifikation för PUM-system},
    colorlinks=true,
    linkcolor=blue,
    filecolor=magenta,      
    urlcolor=cyan,
}

% Definiera formulärfält för testresultat med unika namn
\newcounter{fieldcounter}
\setcounter{fieldcounter}{0}

% Kommando för visuellt resultatfält med unikt ID
\newcommand{\visualfield}[1]{%
\stepcounter{fieldcounter}%
\TextField[name=visual\thefieldcounter,width=1.8cm,height=0.4cm,bordercolor={0.5 0.5 0.5}]{}}

% Kommando för skriptresultatfält med unikt ID
\newcommand{\scriptfield}[1]{%
\stepcounter{fieldcounter}%
\TextField[name=script\thefieldcounter,width=1.8cm,height=0.4cm,bordercolor={0.5 0.5 0.5}]{}}

% Definiera kommando för konsistent tabelllayout i hela dokumentet
\newcommand{\testtabell}{
\begin{longtabu} to \linewidth {|C{1.2cm}|C{1.8cm}|L{4.0cm}|L{7.0cm}|L{6.5cm}|L{6.5cm}|C{2.0cm}|C{2.0cm}|C{2.0cm}|}
}

\begin{document}

\begin{Form}

\maketitle

\section*{Inledning}
Detta dokument innehåller testfall för systemets olika komponenter och funktioner. Varje testfall är utformat för att verifiera en specifik del av systemet. I de fall där testfallet direkt kopplas till ett krav från kravspecifikationen anges kravnumret i kravfältet. För testfall som inte har en direkt koppling till ett specifikt krav i kravspecifikationen har kravfältet lämnats tomt. Testfallen är kategoriserade i följande typer:

\begin{enumerate}
    \item \textbf{Enhetstester} som verifierar att varje komponent (lista, filter, tidslinje, detaljvy) fungerar korrekt var för sig
    \item \textbf{Integrationstester} som kontrollerar att olika komponenter fungerar korrekt tillsammans (lista-filter, tidslinje-filter, etc.)
    \item \textbf{Systemtester} som verifierar systemövergripande funktionalitet som navigering, datahantering och konfiguration
    \item \textbf{Prestandatester} som säkerställer att systemet är snabbt och tillgängligt för alla användare
\end{enumerate}

Varje testfall inkluderar specifik testdata, tydliga exekveringssteg och förväntade resultat för att säkerställa fullständig verifierbarhet av systemets funktioner.

Tabellerna i detta dokument är organiserade för att ge en tydlig översikt över varje testfall med följande information:
\begin{itemize}
    \item \textbf{ID}: En unik identifierare för varje testfall
    \item \textbf{Krav}: Referens till kravet som testas i kravspecifikationen (om tillämpligt)
    \item \textbf{Beskrivning}: Kortfattad beskrivning av funktionaliteten som testas
    \item \textbf{Indata}: Testdata som används för testet
    \item \textbf{Actions}: Steg som ska utföras under testet
    \item \textbf{Förväntat}: Förväntade resultat av testfallet
    \item \textbf{Visuellt}: Interaktivt fält där testresultat för visuell verifiering kan anges
    \item \textbf{Script}: Interaktivt fält där testresultat för automatiserad testning kan anges
    \item \textbf{Implementerat}: Status för implementering av testfallet
\end{itemize}

Detta format har valts för att maximera läsbarheten och ge en omfattande bild av testfallens omfattning och täckning. Dokumentet är formaterat i liggande A3-format för att rymma all nödvändig information på ett läsbart sätt.

\tableofcontents

\newpage

\section{Enhetstester}

\subsection{Lista (L1-L11)}

\testtabell
\hline
\rowcolor{unitcolor}
\textbf{\small ID} & \textbf{\small Krav} & \textbf{\small Beskrivning} & \textbf{\small Indata} & \textbf{\small Actions} & \textbf{\small Förväntat} & \textbf{\small Visuellt} & \textbf{\small Script} & \textbf{\small Imp.} \\
\hline
\endhead

L1 & K1.1-1 & Dokumentlista ska finnas & 
API-svar från /api/journals med journaler: \{\{CompositionId: '1-3', DateTime: '2024-01-01T12:00:00', DisplayDateTime: '2024-01-01 12:00', Dokument\_ID: 'DOC001-003', Dokumentnamn: 'Läkaranteckning 1-3', Vårdenhet\_Namn: 'Kardiologiska kliniken', etc.\}\} & 
Navigera till basvyn & 
Dokumentlistan är synlig & 
\visualfield{} & \scriptfield{} & \\
\hline

L2 & & Kronologisk ordning & 
Journaler med datum: \{\{CompositionId: '1', DateTime: '2023-12-30T12:00:00', Dokumentnamn: 'Läkaranteckning 1'\}, \{CompositionId: '2', DateTime: '2023-12-31T12:00:00', Dokumentnamn: 'Middle Document'\}, \{CompositionId: '3', DateTime: '2024-01-01T12:00:00', Dokumentnamn: 'Läkaranteckning 2'\}\} & 
Navigera till basvyn, expandera alla vårdenhetsgrupper & 
Journaler visas sorterade med senaste först (Läkaranteckning 2 överst, därefter Middle Document, sist Läkaranteckning 1) & 
\visualfield{} & \scriptfield{} & \\
\hline

L3 & & Metadata-visning & 
Journaler med fullständig metadata: \{\{CompositionId: '1', DateTime: '2023-12-30T12:00:00', DisplayDateTime: '2023-12-30 12:00', Dokument\_ID: 'DOC001', Dokument\_skapad\_av\_yrkestitel\_ID: 'L001', Dokument\_skapad\_av\_yrkestitel\_Namn: 'Läkare', Dokumentationskod: 'Läkaranteckning', Dokumentnamn: 'Läkaranteckning 1', Vårdenhet\_Identifierare: 'KK001', Vårdenhet\_Namn: 'Kardiologiska kliniken'\}\} & 
Navigera till basvyn, expandera alla vårdenhetsgrupper & 
All metadata visas för varje journal (Dokumentnamn, Dokumentationskod, yrkesroll, vårdenhet, datum) & 
\visualfield{} & \scriptfield{} & \\
\hline

L4 & & Tom lista & 
Tom array av journaler: [] & 
Navigera till basvyn & 
Listan är tom, inga vårdenhetsgrupper visas & 
\visualfield{} & \scriptfield{} & \\
\hline

L5 & & Gruppering & 
Journaler från olika vårdenheter: \{\{CompositionId: '1', Vårdenhet\_Namn: 'Kardiologiska kliniken'\}, \{CompositionId: '2', Vårdenhet\_Namn: 'Neurologiska avdelningen'\}, \{CompositionId: '3', Vårdenhet\_Namn: 'Onkologiska kliniken'\}, \{CompositionId: '4', Vårdenhet\_Namn: 'Kardiologiska kliniken'\}\} & 
Navigera till basvyn & 
Journalerna visas grupperade i 3 grupper efter vårdenhet (Kardiologiska kliniken, Neurologiska avdelningen, Onkologiska kliniken) & 
\visualfield{} & \scriptfield{} & \\
\hline

L6 & & Kollapsa grupp & 
Journaler från minst en vårdenhet: \{\{CompositionId: '1', Vårdenhet\_Namn: 'Kardiologiska kliniken'\}, \{CompositionId: '2', Vårdenhet\_Namn: 'Kardiologiska kliniken'\}\} & 
Expandera vårdenhetsgruppen 'Kardiologiska kliniken', klicka på grupprubriken & 
Gruppen visar endast rubrik och antalsindikator (2) men inga journaler & 
\visualfield{} & \scriptfield{} & \\
\hline

L7 & & Expandera grupp & 
Journaler från minst en vårdenhet: \{\{CompositionId: '1', Vårdenhet\_Namn: 'Kardiologiska kliniken'\}, \{CompositionId: '2', Vårdenhet\_Namn: 'Kardiologiska kliniken'\}\} & 
Klicka på en kollapsad grupprubrik 'Kardiologiska kliniken' & 
Gruppen expanderas och visar alla journaler under Kardiologiska kliniken & 
\visualfield{} & \scriptfield{} & \\
\hline

L8 & K1.2-3 & Markera journal & 
Minst en journal i listan & 
Klicka på journalen 'Läkaranteckning 1' i listan & 
Journalen markeras med färgad kant och bakgrund, aria-selected='true' & 
\visualfield{} & \scriptfield{} & \\
\hline

L9 & K1.2-3 & Avmarkera journal & 
Minst en journal i listan & 
Markera 'Läkaranteckning 1', klicka utanför listan i body-elementet & 
Journalen är inte längre markerad, aria-selected='false' & 
\visualfield{} & \scriptfield{} & \\
\hline

L10 & K1.2-3 & Markera flera journaler & 
Minst tre journaler i listan & 
Klicka på 'Läkaranteckning 1', håll ned Ctrl/Cmd, klicka på 'Middle Document' och 'Läkaranteckning 2' & 
Alla tre journaler markeras med färgad kant och bakgrund, alla har aria-selected='true' & 
\visualfield{} & \scriptfield{} & \\
\hline

L11 & K1.2-3 & Avmarkera journaler & 
Minst tre journaler i listan & 
Markera tre journaler med Ctrl/Cmd-tangenten, klicka utanför listan i body-elementet & 
Inga journaler är markerade, alla har aria-selected='false' & 
\visualfield{} & \scriptfield{} & \\
\hline

\end{longtabu}

\subsection{Filter (F1-F23)}

\testtabell
\hline
\rowcolor{unitcolor}
\textbf{\small ID} & \textbf{\small Krav} & \textbf{\small Beskrivning} & \textbf{\small Indata} & \textbf{\small Actions} & \textbf{\small Förväntat} & \textbf{\small Visuellt} & \textbf{\small Script} & \textbf{\small Imp.} \\
\hline
\endhead

F1 & K2.1-3 & Panel existerar & 
Mock-data från filter\_tests.py: \{patients: [\{id: 1, name: 'Anna Andersson', personalNumber: '19800101-1234'\}], documents: [\{id: 1, patient\_id: 1, title: 'Läkarbesök', type: 'Journal', category: 'Anteckning', unit: 'Kardiologi', professional: 'Läkare', date: '2024-01-01'\}]\} & 
Navigera till basvyn & 
Filterpanelen (\#Filtermenu) är synlig & 
\visualfield{} & \scriptfield{} & test\_F1\_panel\_exists \\
\hline

F2 & K3.2-3 & Datumväljare finns & 
Mock-data från filter\_tests.py & 
Navigera till basvyn & 
Datum-inmatningsfält (\#DateDiv, \#OldestDate, \#NewestDate) finns och är synliga & 
\visualfield{} & \scriptfield{} & test\_F2\_date\_picker \\
\hline

F3 & K3.2-3 & Mallvals-lista & 
Mock-data med olika journaltyper: \{documents: [\{type: 'Journal'\}, \{type: 'Labbresultat'\}, \{type: 'Röntgensvar'\}]\} & 
Klicka/hovra på mallvalsfältet (\#template) & 
Dropdown med mallar (\#dropdown\_1) visas och innehåller alternativ & 
\visualfield{} & \scriptfield{} & test\_F3\_journal\_type\_list \\
\hline

F4 & K3.2-3 & Vårdenhetslista & 
Mock-data med olika vårdenheter: \{documents: [\{unit: 'Kardiologi'\}, \{unit: 'Laboratorium'\}, \{unit: 'Röntgen'\}]\} & 
Klicka/hovra på vårdenhetsfältet (\#Vårdenhet) & 
Dropdown med vårdenheter (\#dropdown\_2) visas och innehåller alternativ & 
\visualfield{} & \scriptfield{} & test\_F4\_unit\_list \\
\hline

F5 & K3.2-3 & Sökfält & 
Mock-data från filter\_tests.py & 
Navigera till basvyn & 
Sökfält (\#Search input) finns och kan ta emot text 'test search' & 
\visualfield{} & \scriptfield{} & test\_F5\_search\_field \\
\hline

F6 & K3.2-3 & Yrkesrollslista & 
Mock-data med olika yrkesroller: \{documents: [\{professional: 'Läkare'\}, \{professional: 'Sjuksköterska'\}, \{professional: 'Radiolog'\}]\} & 
Klicka/hovra på yrkesrollsfältet (\#role) & 
Dropdown med yrkesroller (\#dropdown\_3) visas och innehåller alternativ & 
\visualfield{} & \scriptfield{} & test\_F6\_professional\_role\_list \\
\hline

F7 & K3.2-10 & Återställningsknapp & 
Mock-data från filter\_tests.py & 
Navigera till basvyn & 
Återställningsknapp (\#Reset) finns och är synlig & 
\visualfield{} & \scriptfield{} & test\_F7\_reset\_button \\
\hline

F8 & K3.2-4 & Datumfiltrering & 
Mock-data med journaler från olika datum: \{documents: [\{date: today\}, \{date: yesterday\}, \{date: last\_week\}, \{date: last\_week\}]\} & 
Ange datum i från- och till-fälten (från yesterday till today) & 
Filtrering uppdateras och visar endast journaler från valda datum (2 av 4) & 
\visualfield{} & \scriptfield{} & test\_F8\_date\_filtering \\
\hline

F9 & K3.2-5 & Mallvals-filtrering & 
Mock-data med olika dokumenttyper: \{documents: [\{type: 'Journal', title: 'Läkarbesök'\}, \{type: 'Labbresultat', title: 'Provtagning'\}, \{type: 'Röntgensvar', title: 'Röntgen'\}, \{type: 'Journal', title: 'Uppföljning'\}]\} & 
Välj 'Journal' från dropdown (\#dropdown\_1) & 
Filtrering uppdateras och visar endast journaler av typ 'Journal' (2 av 4) & 
\visualfield{} & \scriptfield{} & test\_F9\_journal\_type\_filter \\
\hline

F10 & K3.2-6 & Vårdenhetsfiltrering & 
Mock-data med olika vårdenheter: \{documents: [\{unit: 'Kardiologi', title: 'Läkarbesök'\}, \{unit: 'Laboratorium', title: 'Provtagning'\}, \{unit: 'Röntgen', title: 'Röntgen'\}, \{unit: 'Kardiologi', title: 'Uppföljning'\}]\} & 
Välj 'Kardiologi' från dropdown (\#dropdown\_2) & 
Filtrering uppdateras och visar endast journaler från 'Kardiologi' (2 av 4) & 
\visualfield{} & \scriptfield{} & test\_F10\_unit\_filter \\
\hline

F11 & K3.2-7 & Söktermsfiltrering & 
Mock-data med varierande innehåll: \{documents: [\{title: 'Läkarbesök', content: 'Patienten mår bra'\}, \{title: 'Provtagning', content: 'Alla värden inom normalintervall'\}, \{title: 'Röntgen', content: 'Inga patologiska fynd'\}, \{title: 'Uppföljning', content: 'Patienten svarar bra på behandlingen'\}]\} & 
Ange söktermen 'Patienten' i sökfältet (\#Search input) & 
Filtrering uppdateras och visar endast journaler med termen 'Patienten' (2 av 4) & 
\visualfield{} & \scriptfield{} & test\_F11\_search\_term\_filter \\
\hline

F12 & K3.2-8 & Yrkesrollsfiltrering & 
Mock-data med olika yrkesroller: \{documents: [\{professional: 'Läkare', title: 'Läkarbesök'\}, \{professional: 'Sjuksköterska', title: 'Provtagning'\}, \{professional: 'Radiolog', title: 'Röntgen'\}, \{professional: 'Läkare', title: 'Uppföljning'\}]\} & 
Välj 'Läkare' från dropdown (\#dropdown\_3) & 
Filtrering uppdateras och visar endast journaler från 'Läkare' (2 av 4) & 
\visualfield{} & \scriptfield{} & test\_F12\_professional\_role\_filter \\
\hline

F13 & K3.2-9 & Kombinerade filter & 
Mock-data med varierande datum, typ, enhet och roll: \{documents: [\{date: today, type: 'Journal', unit: 'Kardiologi', professional: 'Läkare', title: 'Läkarbesök'\}, \{date: yesterday, type: 'Labbresultat', unit: 'Laboratorium', professional: 'Sjuksköterska', title: 'Provtagning'\}, \{date: last\_week, type: 'Röntgensvar', unit: 'Röntgen', professional: 'Radiolog', title: 'Röntgen'\}, \{date: last\_week, type: 'Journal', unit: 'Kardiologi', professional: 'Läkare', title: 'Uppföljning'\}]\} & 
Välj datumintervall last\_week till today, välj typ 'Journal', välj enhet 'Kardiologi' & 
Filtrering uppdateras och visar endast journaler som uppfyller alla kriterier (1 av 4) & 
\visualfield{} & \scriptfield{} & test\_F13\_combined\_filters \\
\hline

F14 & K3.2-10 & Återställning av filter & 
Mock-data från filter\_tests.py & 
Ange 'Patienten' i sökfältet, välj 'Läkare' från roller, klicka på återställningsknappen (\#Reset) & 
Alla filter återställs till standardvärden, inga söktermer visas, inga filter är valda & 
\visualfield{} & \scriptfield{} & test\_F14\_reset\_filters \\
\hline

F15 & K2.6 & Hantering av ogiltigt datum & 
Mock-data från filter\_tests.py & 
Försök ange ogiltigt datum i datumfältet (\#OldestDate) & 
Systemet hanterar ogiltigt datum, förhindrar inmatning av ogiltigt datum (HTML date-input validering) & 
\visualfield{} & \scriptfield{} & test\_F15\_invalid\_date \\
\hline

F16 & K2.6 & Tom sökning & 
Mock-data från filter\_tests.py & 
Ange text 'test' i sökfältet, rensa fältet, klicka utanför & 
Systemet hanterar tom sökning, visar alla journaler & 
\visualfield{} & \scriptfield{} & test\_F16\_empty\_search \\
\hline

F17 & K2.6 & Datumordning & 
Mock-data från filter\_tests.py & 
Ange slutdatum (today) före startdatum (yesterday) i datumfälten & 
Systemet hanterar ogiltig datumordning & 
\visualfield{} & \scriptfield{} & test\_F17\_date\_order \\
\hline

F18 & K2.6 & Specialtecken & 
Mock-data från filter\_tests.py & 
Ange specialtecken '!@\#\$\%' i sökfältet (\#Search input) & 
Systemet hanterar specialtecken i sökning & 
\visualfield{} & \scriptfield{} & test\_F18\_special\_characters \\
\hline

F19 & K3.2-11 & Realtidsuppdatering datum & 
Mock-data från filter\_tests.py med datum från olika perioder & 
Ändra datum i datumfälten från yesterday till today & 
Filtrering uppdateras i realtid vid varje datumändring utan att behöva trycka på någon knapp & 
\visualfield{} & \scriptfield{} & test\_F19\_date\_change \\
\hline

F20 & K3.2-11 & Realtidsuppdatering mall & 
Mock-data med olika journaltyper: \{documents: [\{type: 'Journal'\}, \{type: 'Labbresultat'\}, \{type: 'Röntgensvar'\}]\} & 
Välj 'Journal' i mallvalsdropdown, ändra sedan till 'Labbresultat' & 
Filtrering uppdateras i realtid vid varje ändring av mall & 
\visualfield{} & \scriptfield{} & test\_F20\_type\_change \\
\hline

F21 & K3.2-11 & Realtidsuppdatering vårdenhet & 
Mock-data med olika vårdenheter: \{documents: [\{unit: 'Kardiologi'\}, \{unit: 'Laboratorium'\}, \{unit: 'Röntgen'\}]\} & 
Välj 'Kardiologi' i vårdenhetsdropdown, ändra sedan till 'Laboratorium' & 
Filtrering uppdateras i realtid vid varje ändring av vårdenhet & 
\visualfield{} & \scriptfield{} & test\_F21\_unit\_change \\
\hline

F22 & K3.2-11 & Realtidsuppdatering sökterm & 
Mock-data med varierande textinnehåll & 
Skriv 't', 'te', 'tes', 'test' i sökfältet, tecken för tecken & 
Filtrering uppdateras i realtid efter varje tecken som skrivs & 
\visualfield{} & \scriptfield{} & test\_F22\_search\_term\_change \\
\hline

F23 & K3.2-11 & Realtidsuppdatering yrkesroll & 
Mock-data med olika yrkesroller: \{documents: [\{professional: 'Läkare'\}, \{professional: 'Sjuksköterska'\}, \{professional: 'Radiolog'\}]\} & 
Välj 'Läkare' i yrkesrollsdropdown, ändra sedan till 'Sjuksköterska' & 
Filtrering uppdateras i realtid vid varje ändring av yrkesroll & 
\visualfield{} & \scriptfield{} & test\_F23\_role\_change \\
\hline

\end{longtabu}

\subsection{Tidslinje (T1-T12)}

\testtabell
\hline
\rowcolor{unitcolor}
\textbf{\small ID} & \textbf{\small Krav} & \textbf{\small Beskrivning} & \textbf{\small Indata} & \textbf{\small Actions} & \textbf{\small Förväntat} & \textbf{\small Visuellt} & \textbf{\small Script} & \textbf{\small Imp.} \\
\hline
\endhead

T1 & K2.1-1 & Timeline existerar & 
Mock-data från timeline\_tests.py: [\{CompositionId: '1', DateTime: '2023-12-12T12:00:00', DisplayDateTime: '2023-12-12 12:00', Dokument\_ID: 'DOC001', Dokumentnamn: 'Older Note', Vårdenhet\_Namn: 'Kardiologiska kliniken'\}, ...] & 
Navigera till basvyn, säkerställ att tidslinjen är synlig & 
Tidslinje visas och innehåller förväntade element (data-testid='timeline-container', 'year-button', 'date-indicator', 'toggle-timeline-button') & 
\visualfield{} & \scriptfield{} & test\_T1\_timeline\_exists \\
\hline

T2 & K2.1-1 & Datumhierarki & 
Journaler från 4 olika datum (2023-11-12, 2023-12-12, 2023-12-22, 2024-01-01) som i test\_notes-fixture & 
Navigera till basvyn med testdata & 
År-grupper (2023, 2024), månads-grupper (November, December, Januari) och dag-grupper visas i hierarkisk struktur & 
\visualfield{} & \scriptfield{} & tests/timeline\_tests.py \\
\hline

T3 & K2.1-1 & Anteckningsvisning & 
Journaler med innehåll: [\{CompositionId: '1', DateTime: '2023-12-12T12:00:00', DisplayDateTime: '2023-12-12 12:00', Dokumentnamn: 'Older Note'\}, \{CompositionId: '2', DateTime: '2023-12-22T12:00:00', DisplayDateTime: '2023-12-22 12:00', Dokumentnamn: 'Middle Note'\}, \{CompositionId: '3', DateTime: '2024-01-01T12:00:00', DisplayDateTime: '2024-01-01 12:00', Dokumentnamn: 'Recent Note'\}, \{CompositionId: '4', DateTime: '2023-11-12T12:00:00', DisplayDateTime: '2023-11-12 12:00', Dokumentnamn: 'Much Older Note'\}] & 
Navigera till basvyn, expandera grupper för att visa anteckningar & 
Anteckningar visas med korrekt innehåll ('Older Note', 'Middle Note', etc.) & 
\visualfield{} & \scriptfield{} & tests/timeline\_tests.py \\
\hline

T12 & K2.1-4 & Visa/dölj tidslinje & 
Journaler från olika datum som i test\_notes-fixture & 
Klicka på [data-testid='toggle-timeline-button'] & 
Tidslinje visas/döljs, knapp ändrar text mellan 'Show Timeline' och 'Hide Timeline' & 
\visualfield{} & \scriptfield{} & tests/timeline\_tests.py \\
\hline

\end{longtabu}

\subsection{Detaljvy (D1-D3, DV1)}

\testtabell
\hline
\rowcolor{unitcolor}
\textbf{\small ID} & \textbf{\small Krav} & \textbf{\small Beskrivning} & \textbf{\small Indata} & \textbf{\small Actions} & \textbf{\small Förväntat} & \textbf{\small Visuellt} & \textbf{\small Script} & \textbf{\small Imp.} \\
\hline
\endhead

D1 & & Detaljyn existerar & 
Mockad journaldata med minst en journal: [\{CompositionId: '1', Dokumentnamn: 'Test Journal', DocumentContent: 'Innehåll i journalen'\}] & 
Navigera till basvyn & 
Detaljvyn finns och är synlig (tom innan någon journal är markerad) & 
\visualfield{} & \scriptfield{} & \\
\hline

D2 & & Visa journal & 
Mockad journal med fullständigt innehåll: \{CompositionId: '1', DateTime: '2023-12-15T12:00:00', DisplayDateTime: '2023-12-15 12:00', Dokument\_ID: 'DOC001', Dokument\_skapad\_av\_yrkestitel\_Namn: 'Läkare', Dokumentationskod: 'Läkaranteckning', Dokumentnamn: 'Läkaranteckning om hypertoni', Vårdenhet\_Namn: 'Kardiologiska kliniken', DocumentContent: 'Patienten har haft förhöjt blodtryck sedan senaste besöket. Förskrivit blodtryckssänkande medicin.'\} & 
Markera journalen 'Läkaranteckning om hypertoni' i listan & 
Journalen visas i detaljvyn med korrekt rubrik, metadata och innehåll & 
\visualfield{} & \scriptfield{} & \\
\hline

D3 & & Visa flera journaler & 
Minst tre mockade journaler: [\{CompositionId: '1', Dokumentnamn: 'Journal 1', DocumentContent: 'Innehåll 1'\}, \{CompositionId: '2', Dokumentnamn: 'Journal 2', DocumentContent: 'Innehåll 2'\}, \{CompositionId: '3', Dokumentnamn: 'Journal 3', DocumentContent: 'Innehåll 3'\}] & 
Markera alla tre journaler med Ctrl/Cmd + klick på var och en & 
Alla tre markerade journaler visas i detaljvyn, var för sig & 
\visualfield{} & \scriptfield{} & \\
\hline

DV1 & K1.1-2 & Detaljerad visning & Mockad journaldata med fullständigt innehåll: \{CompositionId: '1', DateTime: '2023-12-15T12:00:00', Dokumentnamn: 'Provtagningssvar', DocumentContent: 'Proverna visar normala värden inom referensintervallen.'\} & Navigera till basvyn, markera journalen & Detaljerad vy visar den valda journalen med fullständigt innehåll & \visualfield{} & \scriptfield{} & \\
\hline

DV2 & K1.1-3 & Filter i basvy & Mockad journaldata med minst 5 journaler med varierande metadata & Navigera till basvyn & Filterpanel finns och är tillgänglig i basvyn & \visualfield{} & \scriptfield{} & \\
\hline

DV3 & K1.1-4 & Basvy existerar & Mockad journaldata & Öppna applikationen & Basvy visas som standardvy med dokumentlista, detaljvy och filterpanel & \visualfield{} & \scriptfield{} & \\
\hline

DV4 & K1.2-1 & Justera JA dynamik & Mockad journaldata, flera markerade journaler & Markera flera journaler, använd inställningar för visningsläge (t.ex. knapp för kompakt visning) & Användaren kan ändra hur markerade journaler visas (t.ex. växla mellan kompakt och expanderad vy) & \visualfield{} & \scriptfield{} & \\
\hline

DV5 & K1.2-2 & Visa markerade JA dynamiskt & Mockad journaldata, flera markerade journaler med långt innehåll & Markera flera journaler med olika längd på innehåll & Journalernas visning anpassas automatiskt för optimal skärmyta, t.ex. med scrollbars eller anpassad höjd & \visualfield{} & \scriptfield{} & \\
\hline

DV6 & K1.2-3 & Markera JA & Mockad journaldata med minst 5 journaler & Klicka på en journal i listan, därefter Ctrl/Cmd + klick på en annan & Journalerna markeras och visas i detaljvy & \visualfield{} & \scriptfield{} & \\
\hline

DV7 & K1.2-5 & Visa i lista & Mockad journaldata med olika datum: [{DateTime: '2023-12-01'}, {DateTime: '2023-12-15'}, {DateTime: '2024-01-05'}] & Navigera till basvyn & Journalerna presenteras i kronologisk ordning i listvyn & \visualfield{} & \scriptfield{} & \\
\hline

\end{longtabu}

\newpage

\section{Integrationstester}

\subsection{Lista-Filter (LF1-LF8)}

\testtabell
\hline
\rowcolor{integrationcolor}
\textbf{\small ID} & \textbf{\small Krav} & \textbf{\small Beskrivning} & \textbf{\small Indata} & \textbf{\small Actions} & \textbf{\small Förväntat} & \textbf{\small Visuellt} & \textbf{\small Script} & \textbf{\small Imp.} \\
\hline
\endhead

LF1 & K3.2-4 & Filtrering på tid & 
Journaler med specifika datum: [\{CompositionId: '1', DateTime: '2021-01-15T12:00:00', DisplayDateTime: '2021-01-15 12:00', Dokumentnamn: 'Journal 1'\}, \{CompositionId: '2', DateTime: '2021-07-15T12:00:00', DisplayDateTime: '2021-07-15 12:00', Dokumentnamn: 'Journal 2'\}, \{CompositionId: '3', DateTime: '2022-01-15T12:00:00', DisplayDateTime: '2022-01-15 12:00', Dokumentnamn: 'Journal 3'\}, \{CompositionId: '4', DateTime: '2020-06-15T12:00:00', DisplayDateTime: '2020-06-15 12:00', Dokumentnamn: 'Journal 4'\}, \{CompositionId: '5', DateTime: '2020-12-15T12:00:00', DisplayDateTime: '2020-12-15 12:00', Dokumentnamn: 'Journal 5'\}, \{CompositionId: '6', DateTime: '2019-01-15T12:00:00', DisplayDateTime: '2019-01-15 12:00', Dokumentnamn: 'Journal 6'\}, \{CompositionId: '7', DateTime: '2022-12-15T12:00:00', DisplayDateTime: '2022-12-15 12:00', Dokumentnamn: 'Journal 7'\}, \{CompositionId: '8', DateTime: '2023-01-15T12:00:00', DisplayDateTime: '2023-01-15 12:00', Dokumentnamn: 'Journal 8'\}] & 
Ange datumfilter 2021-01-01 till 2022-02-02 i datumfälten & 
Endast journaler 1, 2 och 3 visas i listan (datum inom intervallet 2021-01-01 till 2022-02-02) & 
\visualfield{} & \scriptfield{} & \\
\hline

LF2 & K3.2-5 & Filtrering efter journaltyp & 
Journaler av olika typer: [\{CompositionId: '1', Dokumentationskod: 'Läkaranteckning', Dokumentnamn: 'Läkaranteckning 1'\}, \{CompositionId: '2', Dokumentationskod: 'Läkaranteckning', Dokumentnamn: 'Läkaranteckning 2'\}, \{CompositionId: '3', Dokumentationskod: 'Läkaranteckning', Dokumentnamn: 'Läkaranteckning 3'\}, \{CompositionId: '4', Dokumentationskod: 'Labbresultat', Dokumentnamn: 'Blodprov 1'\}, \{CompositionId: '5', Dokumentationskod: 'Labbresultat', Dokumentnamn: 'Blodprov 2'\}, \{CompositionId: '6', Dokumentationskod: 'Remissvar', Dokumentnamn: 'Remissvar 1'\}, \{CompositionId: '7', Dokumentationskod: 'Remissvar', Dokumentnamn: 'Remissvar 2'\}] & 
Välj journaltyp 'Läkaranteckning' från dropdown & 
Endast Läkaranteckning 1, 2 och 3 visas i listan & 
\visualfield{} & \scriptfield{} & \\
\hline

LF3 & K3.2-6 & Filtrera efter vårdenhet/avdelning & 
Journaler från olika vårdenheter: [\{CompositionId: '1', Vårdenhet\_Namn: 'Kardiologiska kliniken', Dokumentnamn: 'Kardio 1'\}, \{CompositionId: '2', Vårdenhet\_Namn: 'Kardiologiska kliniken', Dokumentnamn: 'Kardio 2'\}, \{CompositionId: '3', Vårdenhet\_Namn: 'Kardiologiska kliniken', Dokumentnamn: 'Kardio 3'\}, \{CompositionId: '4', Vårdenhet\_Namn: 'Neurologiska avdelningen', Dokumentnamn: 'Neuro 1'\}, \{CompositionId: '5', Vårdenhet\_Namn: 'Neurologiska avdelningen', Dokumentnamn: 'Neuro 2'\}, \{CompositionId: '6', Vårdenhet\_Namn: 'Onkologiska kliniken', Dokumentnamn: 'Onko 1'\}, \{CompositionId: '7', Vårdenhet\_Namn: 'Onkologiska kliniken', Dokumentnamn: 'Onko 2'\}] & 
Välj vårdenhet 'Kardiologiska kliniken' från dropdown & 
Endast Kardio 1, 2 och 3 visas i listan under en vårdenhetsgrupp & 
\visualfield{} & \scriptfield{} & \\
\hline

LF4 & K3.2-7 & Filtrera på sökord & 
Journaler med olika textinnehåll: [\{CompositionId: '1', Dokumentnamn: 'Journal 1', DocumentContent: 'Patienten har hypertoni sedan flera år tillbaka'\}, \{CompositionId: '2', Dokumentnamn: 'Journal 2', DocumentContent: 'Uppföljning av hypertoni'\}, \{CompositionId: '3', Dokumentnamn: 'Journal 3', DocumentContent: 'Kontroll av blodsocker'\}, \{CompositionId: '4', Dokumentnamn: 'Journal 4', DocumentContent: 'Patienten upplever yrsel'\}, \{CompositionId: '5', Dokumentnamn: 'Journal 5', DocumentContent: 'Årlig hälsokontroll'\}, \{CompositionId: '6', Dokumentnamn: 'Journal 6', DocumentContent: 'Uppföljning av operation'\}, \{CompositionId: '7', Dokumentnamn: 'Journal 7', DocumentContent: 'Patienten har lågt blodtryck'\}] & 
Ange sökordet 'hypertoni' i sökfältet & 
Endast Journal 1 och 2 visas i listan (innehåller ordet 'hypertoni') & 
\visualfield{} & \scriptfield{} & \\
\hline

LF5 & K3.2-8 & Filtrera efter yrkesroll & 
Journaler skapade av olika yrkesroller: [\{CompositionId: '1', Dokument\_skapad\_av\_yrkestitel\_Namn: 'Läkare', Dokumentnamn: 'Journal L1'\}, \{CompositionId: '2', Dokument\_skapad\_av\_yrkestitel\_Namn: 'Läkare', Dokumentnamn: 'Journal L2'\}, \{CompositionId: '3', Dokument\_skapad\_av\_yrkestitel\_Namn: 'Läkare', Dokumentnamn: 'Journal L3'\}, \{CompositionId: '4', Dokument\_skapad\_av\_yrkestitel\_Namn: 'Läkare', Dokumentnamn: 'Journal L4'\}, \{CompositionId: '5', Dokument\_skapad\_av\_yrkestitel\_Namn: 'Sjuksköterska', Dokumentnamn: 'Journal S1'\}, \{CompositionId: '6', Dokument\_skapad\_av\_yrkestitel\_Namn: 'Sjuksköterska', Dokumentnamn: 'Journal S2'\}, \{CompositionId: '7', Dokument\_skapad\_av\_yrkestitel\_Namn: 'Fysioterapeut', Dokumentnamn: 'Journal F1'\}] & 
Välj yrkesroll 'Läkare' från dropdown & 
Endast Journal L1, L2, L3 och L4 visas i listan & 
\visualfield{} & \scriptfield{} & \\
\hline

LF6 & K3.2-9 & Kombination av filter & 
Journaler med kombinerad metadata: [\{CompositionId: '1', DateTime: '2023-01-15T12:00:00', Dokumentationskod: 'Läkaranteckning', Vårdenhet\_Namn: 'Kardiologiska kliniken', Dokumentnamn: 'Kardio L1'\}, \{CompositionId: '2', DateTime: '2023-06-15T12:00:00', Dokumentationskod: 'Läkaranteckning', Vårdenhet\_Namn: 'Kardiologiska kliniken', Dokumentnamn: 'Kardio L2'\}, \{CompositionId: '3', DateTime: '2022-12-15T12:00:00', Dokumentationskod: 'Läkaranteckning', Vårdenhet\_Namn: 'Neurologiska avdelningen', Dokumentnamn: 'Neuro L1'\}, \{CompositionId: '4', DateTime: '2023-03-15T12:00:00', Dokumentationskod: 'Labbresultat', Vårdenhet\_Namn: 'Kardiologiska kliniken', Dokumentnamn: 'Kardio Lab1'\}, \{CompositionId: '5', DateTime: '2022-06-15T12:00:00', Dokumentationskod: 'Läkaranteckning', Vårdenhet\_Namn: 'Kardiologiska kliniken', Dokumentnamn: 'Kardio L-old'\}, \{CompositionId: '6', DateTime: '2024-01-15T12:00:00', Dokumentationskod: 'Läkaranteckning', Vårdenhet\_Namn: 'Kardiologiska kliniken', Dokumentnamn: 'Kardio L-new'\}] & 
Välj datumintervall 2023-01-01 till 2023-12-31, journaltyp 'Läkaranteckning' och vårdenhet 'Kardiologiska kliniken' & 
Endast Kardio L1 och L2 visas i listan (uppfyller alla filterkriterierna) & 
\visualfield{} & \scriptfield{} & \\
\hline

LF7 & K3.2-10 & Återställa filter & 
Mockad journaldata, flera aktiva filter (datum, journaltyp, vårdenhet) & 
Använd flera filter, klicka på återställningsknappen & 
Alla filter återställs till standardinställningar med ett knapptryck & 
\visualfield{} & \scriptfield{} & \\
\hline

LF8 & K3.2-11 & Dynamisk uppdatering av filter & 
Journaler med olika metadata som i LF6 & 
Ändra datumfilter, journaltyp och vårdenhet i snabb följd utan att bekräfta & 
Listan uppdateras direkt efter varje ändring utan att man behöver klicka på 'sök' eller liknande & 
\visualfield{} & \scriptfield{} & \\
\hline

\end{longtabu}

\subsection{Tidslinje-Filter (TF1-TF3)}

\testtabell
\hline
\rowcolor{integrationcolor}
\textbf{\small ID} & \textbf{\small Krav} & \textbf{\small Beskrivning} & \textbf{\small Indata} & \textbf{\small Actions} & \textbf{\small Förväntat} & \textbf{\small Visuellt} & \textbf{\small Script} & \textbf{\small Imp.} \\
\hline
\endhead

TF1 & K2.1-3 & Filter i tidslinje & 
Mockad journaldata med journaler från olika datum och typer & 
Navigera till tidslinjevyn & 
Filterpanel finns och är tillgänglig i tidslinjevyn & 
\visualfield{} & \scriptfield{} & \\
\hline

TF2 & K3.2-3 & Filtrering påverkar tidslinje & 
Mockad journaldata med olika dokumenttyper: journals, labbsvar och röntgensvar & 
Navigera till tidslinjevyn, filtrera på dokumenttyp "labbsvar" & 
Tidslinjen visar endast journaler av typen "labbsvar" & 
\visualfield{} & \scriptfield{} & \\
\hline

TF3 & K3.2-9 & Kombination av filter i tidslinje & 
Journaler med kombinerad metadata: datum, typ, vårdenhet & 
Välj datumintervall, journaltyp och vårdenhet i filterpanelen i tidslinjevyn & 
Tidslinjen uppdateras och visar endast journaler som uppfyller alla filterkriterierna & 
\visualfield{} & \scriptfield{} & \\
\hline

\end{longtabu}

\subsection{Lista-Detaljvy (LD1-LD2)}

\testtabell
\hline
\rowcolor{integrationcolor}
\textbf{\small ID} & \textbf{\small Krav} & \textbf{\small Beskrivning} & \textbf{\small Indata} & \textbf{\small Actions} & \textbf{\small Förväntat} & \textbf{\small Visuellt} & \textbf{\small Script} & \textbf{\small Imp.} \\
\hline
\endhead

LD1 & K1.2-3 & Markera journal visar innehåll & 
Mockad journaldata med fullständigt innehåll: {CompositionId: '1', DateTime: '2023-12-15T12:00:00', DisplayDateTime: '2023-12-15 12:00', Dokumentnamn: 'Läkaranteckning om hypertoni', DocumentContent: 'Patienten har haft förhöjt blodtryck...'} & 
Markera journalen 'Läkaranteckning om hypertoni' i listan & 
Journalen visas i detaljvyn med korrekt rubrik, metadata och innehåll & 
\visualfield{} & \scriptfield{} & \\
\hline

LD2 & K1.2-3 & Flera markerade journaler & 
Minst tre mockade journaler med olika innehåll & 
Markera flera journaler med Ctrl/Cmd + klick på var och en & 
Alla markerade journaler visas i detaljvyn, var för sig & 
\visualfield{} & \scriptfield{} & \\
\hline

\end{longtabu}

\subsection{Tidslinje-Detaljvy (TD1-TD2)}

\testtabell
\hline
\rowcolor{integrationcolor}
\textbf{\small ID} & \textbf{\small Krav} & \textbf{\small Beskrivning} & \textbf{\small Indata} & \textbf{\small Actions} & \textbf{\small Förväntat} & \textbf{\small Visuellt} & \textbf{\small Script} & \textbf{\small Imp.} \\
\hline
\endhead

TD1 & K2.1-1 & Markera i tidslinje visar detaljvy & 
Mockad journaldata med journaler från olika datum & 
Navigera till tidslinjevyn, markera en journal i tidslinjen & 
Den markerade journalen visas i detaljvyn med fullständigt innehåll & 
\visualfield{} & \scriptfield{} & \\
\hline

TD2 & K2.2-1 & Låsa anteckning i tidslinje & 
Mockad journaldata med flera journaler & 
Markera en journal i tidslinjen, aktivera lås-funktionen & 
Journalen visas konstant i detaljvyn även vid navigering i tidslinjen & 
\visualfield{} & \scriptfield{} & \\
\hline

\end{longtabu}

\subsection{Lista-Tidslinje (LT1-LT3)}

\testtabell
\hline
\rowcolor{integrationcolor}
\textbf{\small ID} & \textbf{\small Krav} & \textbf{\small Beskrivning} & \textbf{\small Indata} & \textbf{\small Actions} & \textbf{\small Förväntat} & \textbf{\small Visuellt} & \textbf{\small Script} & \textbf{\small Imp.} \\
\hline
\endhead

LT1 & K3.2-1 & Bevara markerade journaler & 
Mockad journaldata med flera journaler & 
Markera en journal i listan, byt till tidslinjevy & 
Markerade journaler förblir markerade även vid vybyte & 
\visualfield{} & \scriptfield{} & \\
\hline

LT2 & K3.2-16 & Vy navigering & 
Mockad journaldata & 
Klicka på navigationsknappar för att byta mellan basvy och tidslinjevy & 
Användaren kan växla mellan basvy och tidslinjevy & 
\visualfield{} & \scriptfield{} & \\
\hline

LT3 & K3.2-2 & Färgkodning konsistent & 
Mockad journaldata med olika typer av journaler & 
Färgkoda journaler i listan, byt till tidslinjevy & 
Samma färgkodning används i både listvy och tidslinjevy & 
\visualfield{} & \scriptfield{} & \\
\hline

\end{longtabu}

\newpage

\section{Systemtester}

\subsection{Navigering och Vy-växling (S1-S4)}

\testtabell
\hline
\rowcolor{systemcolor}
\textbf{\small ID} & \textbf{\small Krav} & \textbf{\small Beskrivning} & \textbf{\small Indata} & \textbf{\small Actions} & \textbf{\small Förväntat} & \textbf{\small Visuellt} & \textbf{\small Script} & \textbf{\small Imp.} \\
\hline
\endhead

S1 & K1.1-4 & Basvy existerar & 
Mockad journaldata & 
Öppna applikationen & 
Basvy visas som standardvy med dokumentlista, detaljvy och filterpanel & 
\visualfield{} & \scriptfield{} & \\
\hline

S2 & K2.1-4 & Tidslinjevy existerar & 
Mockad journaldata & 
Öppna applikationen, navigera till tidslinjevyn & 
Tidslinjevyn finns och kan nås via navigation & 
\visualfield{} & \scriptfield{} & \\
\hline

S3 & K2.2-2 & Slider/Scrolla i tidslinje & 
Mockad journaldata med journaler från ett större tidsspann (t.ex. 2 år) & 
Navigera till tidslinjevyn, testa horisontal scrollning och/eller slider-kontroll & 
Användaren kan scrolla horisontellt eller använda slider för att navigera i tidslinjen & 
\visualfield{} & \scriptfield{} & \\
\hline

S4 & K1.2-5 & Visa i lista & 
Mockad journaldata med olika datum & 
Navigera till basvyn & 
Journalerna presenteras i kronologisk ordning i listvyn & 
\visualfield{} & \scriptfield{} & \\
\hline

\end{longtabu}

\subsection{Datahantering (S5-S9)}

\testtabell
\hline
\rowcolor{systemcolor}
\textbf{\small ID} & \textbf{\small Krav} & \textbf{\small Beskrivning} & \textbf{\small Indata} & \textbf{\small Actions} & \textbf{\small Förväntat} & \textbf{\small Visuellt} & \textbf{\small Script} & \textbf{\small Imp.} \\
\hline
\endhead

S5 & K3.3-1 & Hämta journaldata & 
Mockad API-endpoint för journaldata & 
Öppna applikationen eller uppdatera data & 
Systemet hämtar journalanteckningar från externa journalsystem via REST-API:er & 
\visualfield{} & \scriptfield{} & \\
\hline

S6 & K3.3-2 & Stöd för openEHR & 
Mockad journaldata i openEHR-format & 
Importera data i openEHR-format, visa data i systemet & 
Systemet kan tolka och visa data i openEHR-standarden & 
\visualfield{} & \scriptfield{} & \\
\hline

S7 & K3.3-3 & Stöd för HL7 FHIR & 
Mockad journaldata i HL7 FHIR-format & 
Importera data i HL7 FHIR-format, visa data i systemet & 
Systemet kan tolka och visa data i HL7 FHIR-standarden & 
\visualfield{} & \scriptfield{} & \\
\hline

S8 & K3.3-4 & Uppdatera journaldata i realtid & 
Mockad API-endpoint som skickar uppdateringar & 
När ny data blir tillgänglig i API:et & 
Systemet uppdaterar journalanteckningar utan att användaren behöver ladda om sidan & 
\visualfield{} & \scriptfield{} & \\
\hline

S9 & K3.3-5 & Hantera fel vid datahämtning & 
Mockad API-endpoint som returnerar felkod (t.ex. 404, 500) & 
Simulera ett API-fel genom att konfigurera mocken att returnera ett fel & 
Systemet visar lämpligt felmeddelande och hanterar felet utan att krascha & 
\visualfield{} & \scriptfield{} & \\
\hline

\end{longtabu}

\subsection{Konfiguration och Inställningar (S10-S13)}

\testtabell
\hline
\rowcolor{systemcolor}
\textbf{\small ID} & \textbf{\small Krav} & \textbf{\small Beskrivning} & \textbf{\small Indata} & \textbf{\small Actions} & \textbf{\small Förväntat} & \textbf{\small Visuellt} & \textbf{\small Script} & \textbf{\small Imp.} \\
\hline
\endhead

S10 & K3.2-12 & Spara filter & 
Mockad journaldata, konfigurerade filter & 
Konfigurera flera filter, klicka på 'Spara filterinställning', ge inställningen ett namn & 
Filterinställningen sparas och kan återanvändas vid senare tillfälle & 
\visualfield{} & \scriptfield{} & \\
\hline

S11 & K3.2-13 & Färgval & 
Mockad journaldata med journaler som innehåller olika nyckelord & 
Öppna färginställningar, välj en kategori/ett nyckelord, välj en ny färg & 
Användaren kan ändra färgen som används vid färgkodning & 
\visualfield{} & \scriptfield{} & \\
\hline

S12 & K3.2-14 & Spara layouts & 
Mockad journaldata & 
Konfigurera layouten (panelstorlekar, vyer, etc.), använd funktion för att spara layout & 
Layouten sparas och kan återanvändas vid senare tillfälle & 
\visualfield{} & \scriptfield{} & \\
\hline

S13 & K3.2-15 & Panelskalning & 
Mockad journaldata & 
Dra i panelkanter/delare för att ändra storlek på dokumentvy/filterpanel/etc. & 
Dokument och sidopaneler kan skalas enligt användarens önskemål & 
\visualfield{} & \scriptfield{} & \\
\hline

\end{longtabu}

\newpage

\section{Prestandatester}

\testtabell
\hline
\rowcolor{performancecolor}
\textbf{\small ID} & \textbf{\small Krav} & \textbf{\small Beskrivning} & \textbf{\small Indata} & \textbf{\small Actions} & \textbf{\small Förväntat} & \textbf{\small Visuellt} & \textbf{\small Script} & \textbf{\small Imp.} \\
\hline
\endhead

P1 & K4-1 & Snabbt gränssnitt & 
Mockad journaldata & 
Ladda applikationen, mät inladdningstid med Google Lighthouse & 
Gränssnittet laddas in på under 1.5 sekunder enligt Google Lighthouse-mätning & 
\visualfield{} & \scriptfield{} & \\
\hline

P2 & K4-2 & Hjälp för färgblinda & 
Mockad journaldata, aktivera färgkodning & 
Inspektera färgerna som används i gränssnittet, särskilt vid färgkodning & 
Gränssnittet använder färgblindvänliga färgskalor vid färgkodning, med tillräcklig kontrast och särskiljande mönster utöver färg & 
\visualfield{} & \scriptfield{} & \\
\hline

\end{longtabu}

\section{Slutsats}

Detta testspecifikationsdokument innehåller en omfattande uppsättning testfall för att verifiera systemets funktionalitet utifrån ett slutanvändarperspektiv. Testfallen är strukturerade i fyra huvudkategorier:

\begin{enumerate}
    \item \textbf{Enhetstester} som verifierar att varje komponent (lista, filter, tidslinje, detaljvy) fungerar korrekt var för sig
    \item \textbf{Integrationstester} som kontrollerar att olika komponenter fungerar korrekt tillsammans (lista-filter, tidslinje-filter, etc.)
    \item \textbf{Systemtester} som verifierar systemövergripande funktionalitet som navigering, datahantering och konfiguration
    \item \textbf{Prestandatester} som säkerställer att systemet är snabbt och tillgängligt för alla användare
\end{enumerate}

Varje testfall inkluderar specifik testdata, tydliga exekveringssteg och förväntade resultat. Testfallen är kopplade till specifika krav från kravspecifikationsdokumentet med exakt samma numreringsformat (KX.Y-Z) för att säkerställa fullständig spårbarhet mellan krav och testning. Testfall som inte kan kopplas direkt till ett specifikt krav i kravspecifikationen har exkluderats från detta dokument.

\section*{Instruktioner för ifyllning}
Detta dokument är interaktivt. Använd följande riktlinjer för ifyllning av testresultat:
\begin{itemize}
    \item \textbf{Visuellt}: Ange "Pass" eller "Fail" baserat på visuell verifiering av testresultatet.
    \item \textbf{Script}: Ange "Pass" eller "Fail" baserat på automatiserad testning, alternativt "N/A" om ingen automatiserad testning utförts.
\end{itemize}
Klicka på de grå fälten och skriv in resultatet. Dokumentet kan sparas med ifyllda resultat för framtida referens.

\end{Form}

\end{document} 