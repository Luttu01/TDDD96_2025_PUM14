\documentclass{article}
\usepackage[utf8]{inputenc}
\usepackage[english, swedish]{babel}
\usepackage{amsmath}
\usepackage{listings}
\usepackage{graphicx}
\usepackage{array}
\usepackage{longtable}
\usepackage{multirow}
\usepackage[a4paper,pdftex,bottom=20mm, width=160mm]{geometry}
\setlength{\parindent}{0pt}
\setlength{\parskip}{1em}
\usepackage{ragged2e}
\newcolumntype{L}[1]{>{\RaggedRight\arraybackslash}p{#1}}

\begin{document}

\title{Enhetstester: Tidslinje (Detaljerad och Översiktlig)}
\author{Kandidatprojekt VT 2025}
\date{\today}

\maketitle

\section{Introduktion}
Detta dokument specificerar enhetstestfall för tidslinjekomponenterna i IVPJ - Detaljerad tidslinje (K2.1-1) och Översiktlig tidslinje (K2.1-2). Testerna fokuserar på isolerad funktionalitet inom dessa komponenter.

\section{Testmiljö}
\subsection{Verktyg}
\begin{itemize}
    \item Pytest med Playwright (eller motsvarande testramverk för Svelte)
    \item Mock-data för journalanteckningar med varierade tidsstämplar
    \item Mockad tidslinjekomponent-status (t.ex. zoomnivå, scrollposition)
\end{itemize}

\section{Testfall}

\subsection{Rendering och Layout - Detaljerad Tidslinje}
\begin{longtable}{|L{1.5cm}|L{3.2cm}|L{4.5cm}|L{4.5cm}|L{2.2cm}|}
\hline
\textbf{Nr} & \textbf{Testfall} & \textbf{Indata (Mockad)} & \textbf{Förväntat resultat} & \textbf{Krav} \\
\hline
1.1 & Render tidslinje & Journaldata med tidsstämplar & Detaljerad tidslinje renderas och visar journaler på en tidsaxel & K2.1-1 \\
\hline
1.2 & Kronologisk ordning & Journaler med olika datum & Journalerna placeras på tidsaxeln i korrekt kronologisk ordning & K2.1-1, K2.2-4 \\
\hline
1.3 & Journalpositionering & Journaler med jämnt fördelade datum & Avståndet mellan journalerna på tidsaxeln är proportionellt mot tid mellan datumen & K2.1-1, K2.2-4 \\
\hline
1.4 & Render tom tidslinje & Ingen journaldata & Visas med tom tidsaxel och meddelande "Inga journaler att visa" & K2.1-1 \\
\hline
1.5 & Tidsperiod & Startdatum 2024-01-01, slutdatum 2024-12-31 & Tidslinjen visar korrekt tidsperiod på axeln & K2.1-1, K3.2-4 \\
\hline
1.6 & Visuell markering & Journaler med specifik status "markerad" & Markerade journaler visas med distinkt visuell indikator på tidslinjen & K2.1-1, K3.2-1 \\
\hline
1.7 & Låsta journaler & Journal med status "låst" & Låsta journaler stannar synliga i detaljerad vy oavsett scrollning & K2.1-1, K2.2-1 \\
\hline
\end{longtable}

\subsection{Rendering och Layout - Översiktlig Tidslinje}
\begin{longtable}{|L{1.5cm}|L{3.2cm}|L{4.5cm}|L{4.5cm}|L{2.2cm}|}
\hline
\textbf{Nr} & \textbf{Testfall} & \textbf{Indata (Mockad)} & \textbf{Förväntat resultat} & \textbf{Krav} \\
\hline
2.1 & Render översikt & Journaldata med tidsstämplar & Översiktlig tidslinje renderas med minimerade journalrepresentationer & K2.1-2, K2.2-5 \\
\hline
2.2 & Kompakt representation & Många journaler inom kort tidsperiod & Journalerna på översiktstidslinjen visas i kompakt format utan överlappning & K2.1-2, K2.2-5 \\
\hline
2.3 & Viewport-indikator & Mockad visningsposition (start- och slutpunkt) & Översiktstidslinjen visar korrekt viewport-indikator som representerar den synliga delen & K2.1-2, K2.2-5 \\
\hline
2.4 & Hela tidsperioden & Journaler över flera år & Översiktstidslinjen visar hela den tillgängliga tidsperioden i komprimerad form & K2.1-2, K2.2-5 \\
\hline
\end{longtable}

\subsection{Interaktion - Detaljerad Tidslinje}
\begin{longtable}{|L{1.5cm}|L{3.2cm}|L{4.5cm}|L{4.5cm}|L{2.2cm}|}
\hline
\textbf{Nr} & \textbf{Testfall} & \textbf{Indata (Mockad)} & \textbf{Förväntat resultat} & \textbf{Krav} \\
\hline
3.1 & Horisontell scrollning & Simulera scroll-event på tidslinjen & Tidslinjen scrollar horisontellt och visar tidigare/senare journaler & K2.2-2 \\
\hline
3.2 & Låsning av journal & Klick på "lås"-ikon för en journal & Journalen markeras som låst och förblir synlig vid scrollning & K2.2-1 \\
\hline
3.3 & Journalselektion & Klick på journal i tidslinjen & Journalen markeras visuellt i tidslinjen och journalens tillstånd uppdateras till "markerad=true" & K2.2-4 \\
\hline
3.4 & Zoom in & Simulera zoom-in-event & Tidslinjen zoomar in och visar färre journaler med mer detalj & K2.2-3 \\
\hline
3.5 & Zoom ut & Simulera zoom-ut-event & Tidslinjen zoomar ut och visar fler journaler med mindre detalj & K2.2-3 \\
\hline
\end{longtable}

\subsection{Interaktion - Översiktlig Tidslinje}
\begin{longtable}{|L{1.5cm}|L{3.2cm}|L{4.5cm}|L{4.5cm}|L{2.2cm}|}
\hline
\textbf{Nr} & \textbf{Testfall} & \textbf{Indata (Mockad)} & \textbf{Förväntat resultat} & \textbf{Krav} \\
\hline
4.1 & Klick på position & Simulera klick på position i översiktstidslinjen & Komponentens interna tillstånd uppdateras med nya fokuspunkten och emit-händelse triggas med korrekt tidsstämpel & K2.2-5, K2.2-2 \\
\hline
4.2 & Viewport-uppdatering & Skicka nytt viewport-intervall till komponenten & Viewport-indikatorn uppdateras visuellt för att visa det nya intervallet & K2.2-5 \\
\hline
4.3 & Dra i slider & Simulera drag-event på översiktstidslinjens slider & Sliderposition uppdateras visuellt och emit-händelse triggas med korrekt positionsdata & K2.2-5, K2.2-2 \\
\hline
\end{longtable}

\subsection{Funktioner}
\begin{longtable}{|L{1.5cm}|L{3.2cm}|L{4.5cm}|L{4.5cm}|L{2.2cm}|}
\hline
\textbf{Nr} & \textbf{Testfall} & \textbf{Indata (Mockad)} & \textbf{Förväntat resultat} & \textbf{Krav} \\
\hline
5.1 & Filtrera journaltyp & Filtrera till endast läkaranteckningar & Tidslinjen visar endast läkaranteckningar, korrekt positionerade & K2.1-3, K3.2-5 \\
\hline
5.2 & Filtrera tidsperiod & Ändra tidsintervall från 1 år till 3 månader & Tidslinjen justeras för att visa endast journaler inom det nya intervallet & K2.1-3, K3.2-4 \\
\hline
5.3 & Filtrera sökord & Sök efter "feber" & Tidslinjen visar endast journaler innehållande "feber" & K2.1-3, K3.2-7 \\
\hline
5.4 & Färgkodning & Aktivera färgkodning för specifika nyckelord & Journaler på tidslinjen visar färgkodning för matchande nyckelord & K3.2-2, K3.2-13 \\
\hline
5.5 & Ladda sparad status & Ladda komponent med sparad tillståndsinformation (markerad och låst journal) & Komponenten renderas med korrekt visuell representation av journalens tillstånd enligt sparad data & K3.2-1 \\
\hline
\end{longtable}

\end{document} 